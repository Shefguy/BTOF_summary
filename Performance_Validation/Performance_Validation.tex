\documentclass[../BTOF_summary.tex]{subfiles}
 
\begin{document}

\section{Performance Validation}

\subsection{Time Resolution Surface Scans}
The following scans were performed at the University of Erlangen in collaboration with the group of A. Lehmann.
A more detailed discussion of these measurements can be found in the dissertation of Sebastian Zimmermann\footnote{\url{https://panda.gsi.de/publication/th-phd-2021-001}}.

To ensure a performance as homogeneous as possible over the entire scintillator surface the time resolution was measured at multiple points along the scintillator surface.
This particular test was performed to evaluate the ideal scintillator thickness which, as shown in \fig~\ref{fig:Tchickness_timeRes}, was determined to be \SI{5}{mm}.

The measurements were performed using a \sr\ source and a small trigger scintillator on a motorized arm, reading out the \sipm\ arrays on the two short sides of the scintillator, to determine the time resolution of the detector module across the entire scintillator surface.
The relevant measurement for a \SI{5}{mm} thick scintillator is shown in \fig~\ref{fig:Time_res_scan_erlangen}.

\begin{figure}[htbp]
    \centering
    \includegraphics*[width=.9\textwidth]{fig/TimeResolution_5mm.pdf}
    \caption{Time resolution surface scan of a \SI{5}{mm} thick scintillator tile.}
    \label{fig:Time_res_scan_erlangen}
\end{figure}

\begin{figure}
    \centering
    \includegraphics*[width=.7\textwidth]{fig/run52_widthfit.pdf}
    \caption{Histogram of the time resolution over the entire surface, fit with a gaussian distribution.}
    \label{fig:timeRes_scan_fit}
\end{figure}
    

\end{document}