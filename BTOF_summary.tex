\documentclass[12pt,a4paper,oneside]{article}

\usepackage[utf8]{inputenc}
\usepackage[english]{babel}
\usepackage{subfiles}			% for multi-document support
\usepackage{amsmath}
\usepackage{amsfonts}
\usepackage{amssymb}
\usepackage{hyperref}

\usepackage[pdftex,dvipsnames]{xcolor}  % Coloured text etc.
\usepackage{enumerate}% http://ctan.org/pkg/enumerate
\usepackage[separate-uncertainty=true, exponent-product = \cdot, product-units = power]{siunitx}
\sisetup{detect-all}
\usepackage{hyperref}
\hypersetup{
	colorlinks,
	linkcolor = {blue}
}
\usepackage[font=small,format=plain,labelfont=bf,up,textfont=it,up]{caption}
\usepackage[inline, shortlabels]{enumitem}	% make enumerate environment write in a single line

\usepackage{graphicx}
\graphicspath{{./Bilder/}}
\usepackage[left=2cm,right=3cm,top=2cm,bottom=2cm]{geometry}
\usepackage[perpage]{footmisc}
\usepackage{booktabs}
\usepackage{xargs}                      % Use more than one optional parameter in a new commands
\setlength {\marginparwidth }{2.6cm}
\usepackage[colorinlistoftodos,prependcaption,textsize=tiny,textwidth=\linewidth/1.0]{todonotes}
\newcommandx{\unsure}[2][1=]{\todo[linecolor=red,backgroundcolor=red!25,bordercolor=red,#1]{#2}}
\newcommandx{\todosvetlana}[2][1=]{\todo[linecolor=blue,backgroundcolor=blue!25,bordercolor=blue,#1]{#2}}
\newcommandx{\todosebastian}[2][1=]{\todo[linecolor=YellowOrange,backgroundcolor=YellowOrange!25,bordercolor=YellowOrange,#1]{#2}}
\newcommandx{\info}[2][1=]{\todo[linecolor=OliveGreen,backgroundcolor=OliveGreen!25,bordercolor=OliveGreen,#1]{#2}}
\newcommandx{\improvement}[2][1=]{\todo[linecolor=Plum,backgroundcolor=Plum!25,bordercolor=Plum,#1]{#2}}
\newcommandx{\thiswillnotshow}[2][1=]{\todo[disable,#1]{#2}}

\newcommand{\fig}{Fig.}
\newcommand{\panda}{\textsc{$\overline{\textsc{P}}$anda}}
\newcommand{\PANDA}{\panda}
\newcommand{\fair}{FAIR}
\newcommand{\btof}{B-ToF}
\newcommand{\btofD}{B-ToF detector}
\newcommand{\SciTil}{\btof}
\newcommand{\btofLong}{Barrel Time-of-Flight Detector}
\newcommand{\btofLongD}{Barrel Time-of-Flight Detektor}
\newcommand{\bdirc}{Barrel DIRC}
\newcommand{\ddirc}{Disc DIRC}
\newcommand{\sm}{Super-Module}
\newcommand{\sensorboard}{Sensor-Board}
\newcommand{\railboard}{Rail-Board}
\newcommand{\hamamatsu}{Hamamatsu}
\newcommand{\ketek}{Ketek}
\newcommand{\advansid}{AdvanSiD}
\newcommand{\antiproton}{anti-proton}
\newcommand{\sipm}{SiPM}
\newcommand{\sipms}{SiPM's}
\newcommand{\proot}{PandaRoot}
\newcommand{\sr}{$^{90}$Sr}

\author{Sebastian Zimmermann, Svetlana Chesnevskya}
\title{The Barrel Time-Of-Flight Detector}

\begin{document}

\maketitle

\begin{figure}[h]
	\centering
	\includegraphics[width=\textwidth]{BTOF_detector2.png}
\end{figure}

\section*{Aim of the Document}
The aim of this document is to give a broad overview of the detector summarizing and expanding on the established Technical Design Report written by K. Suzuki et al.

\newpage
\tableofcontents
\newpage

%%%%%%%%%%%%%%%%%%%%%%%%%%%%%%%%%%%%%%%%%%%%%%%%%%%%%%%%%%%%%%%%%%%%%%
% 					section Introduction
\subfile{./Introduction/Introduction.tex}


%%%%%%%%%%%%%%%%%%%%%%%%%%%%%%%%%%%%%%%%%%%%%%%%%%%%%%%%%%%%%%%%%%%%%%
\subfile{./Detector_Hardware/Detector_Hardware.tex}
\newpage
%%%%%%%%%%%%%%%%%%%%%%%%%%%%%%%%%%%%%%%%%%%%%%%%%%%%%%%%%%%%%%%%%%%%%%
\subfile{./Infrastructure/Infrastructure.tex}
\newpage
%%%%%%%%%%%%%%%%%%%%%%%%%%%%%%%%%%%%%%%%%%%%%%%%%%%%%%%%%%%%%%%%%%%%%%
\subfile{./Capabilities/Capabilities.tex}
\newpage
%%%%%%%%%%%%%%%%%%%%%%%%%%%%%%%%%%%%%%%%%%%%%%%%%%%%%%%%%%%%%%%%%%%%%%
\subfile{./Performance_Validation/Performance_Validation.tex}
\newpage
%%%%%%%%%%%%%%%%%%%%%%%%%%%%%%%%%%%%%%%%%%%%%%%%%%%%%%%%%%%%%%%%%%%%%%
\subfile{./Calibration/Calibration.tex}
\newpage
%%%%%%%%%%%%%%%%%%%%%%%%%%%%%%%%%%%%%%%%%%%%%%%%%%%%%%%%%%%%%%%%%%%%%%
\subfile{./Readout/Readout.tex}
\newpage

%%%%%%%%%%%%%%%%%%%%%%%%%%%%%%%%%%%%%%%%%%%%%%%%%%%%%%%%%%%%%%%%%%%%%%
\subfile{./VendorProduction/VendorProduction.tex}
\newpage

%%%%%%%%%%%%%%%%%%%%%%%%%%%%%%%%%%%%%%%%%%%%%%%%%%%%%%%%%%%%%%%%%%%%%%
\subfile{./Open_Challenges/Open_Challenges}


\newpage
\todo[inline]{adjust the margins back to a symmetrical layout}
%\listoftodos


















\end{document}
