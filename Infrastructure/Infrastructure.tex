\documentclass[../BTOF_summary.tex]{subfiles}
 
\begin{document}
\section{Infrastructure}

\subsection{Power Supply}

The two main devices that need to be powered are the FEE and the \sipms .
The FEE as they are foreseen at the moment require a \SI{12}{V} supply.
The needs of the \sipms\ depend on the chosen \sipm\ model. Since four \sipms\ are connected in series and the necessary applied voltage adds up, small differences in the operational voltage of individual \sipms\ can change the requirements on the power supply.

In case \hamamatsu\ \sipms\ with an operational voltage of \SI{\sim 60}{V} are used the power supply is required to deliver voltages upwards of \SI{240}{V}.

The TOFPET ASIC system that is foreseen as the base of the FEE is equipped with an internal power supply however only capable of delivering voltages up to around \SI{90}{V}.
As shown this might not be sufficient.
In this case an external power source is required.

The performance of the \sipms\ is also dependant on the applied overvoltage\unsure{does this need explaining?}.
Since the manufacturing of the \sipms\ is not perfect slight variations in the breakdown voltages of every \sipm\ are to be expected.
These variations lead to differences in the applied overvoltage and would need to be adjusted for to ensure identical performance of every detector module, meaning individual power supply channels for every readout channel.
This however is not feasible for the 3840 channels of the detector.

Alternatively multiple detector module can be connected to the same biasing line.
Measurements done in Erlangen with \hamamatsu\ \texttt{S13360-3050PE} \sipms\ to determine the optimal bias voltage for each side of the detector showed a slight performance dependance.
If the \sipms\ are pre-sorted and grouped by breakdown voltage the overvoltage mismatch can be minimized.

\subsection{Light Tight Enclosure}

To ensure minimal noise from stray photons the detector super-modules need to be packed in a light tight enclosure.
Similar to the \bdirc\ it is foreseen that the carbon holding frame will act as the photon barrier.

\subsection{Cooling}

The active electrical components of the detector can be separated into two main regions with different cooling requirements.
The first region is behind the interaction point at the end of the \sm\ where the Front-End-Electronics are positioned.
Most cooling is required here.
The second region is the \SI{1.8}{m} long section with all the optical and temperature sensors, as well as the LED's.

\subsubsection*{Cooling of FEE}

The main active component of the \btofD\ is the FEE placed at the end of the detector in front of the interaction point.
%As the main heat source of the detector it is foreseen to be water cooled.
%
Considering the standard readout system produced by PETsys we can expect the following heat output.
Each channel of the TOFPET2 ASIC by PETsys dissipates \SI{8.2}{mW} of heat plus \SI{6}{mW} per channel for the LVDS buffers and LDO voltage regulators, which might not be required in the final version when the power supply is separated from the FEE.
This brings the total heat output to \SI{14.2}{mW} per channel.
For the 240 channels of the \sm\ this adds up to \SI{3.4}{W} of heat energy.
In addition to that the TOFPET2 ASIC's need to be read out by FEB/D of the PETsys system, of which one is required per \sm .

Each \sm\ can be equipped with up to one FEB/D, which needs a power supply capable of providing \SI{12}{V} and \SI{4}{A}, which results in a maximum power output of \SI{48}{W}.
In total this means up to \SI{51.4}{W} of power need to be cooled.

Depending on the available space, which depends on the final design of the FEE and the availability of cooling solutions either water or air cooling can be considered.
In general a \SI{50}{W} heat source can easily be cooled with a passive cooler.

\subsubsection*{Cooling of Sensors}

Although there is a minimal amount of electrical components placed along the \railboard\ some heat will still be generated; mainly by the \sipms .
Since the \sipm\ performance also is very dependant on the temperature a slight draft of pressured air along the \railboard\ should keep everything dry and produce a constant temperature.

\end{document}