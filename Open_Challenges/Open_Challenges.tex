
\documentclass[../BTOF_summary.tex]{subfiles}
 
\begin{document}


\section{Open Challenges}

Things that still need to be completed:

\begin{enumerate}
    \item Readout Electronics (TOFPET ASIC)
    \item Temperature sensor control 
    \item Calibration LED control
    \item Decision making on split / complete \railboard
\end{enumerate}

\subsection{Calibration LED}

In order to provide ongoing performance monitoring LED's on the sensor boards are meant to be flashed.
The resulting signal can be used to determine the time resolution or potential \sipm\ gain changes.
The optimal electric pulse to produce the ideal LED signal was subject to investigation as part of a masters thesis.
This project however was never finished and the question of the ideal LED driver remains to be determined.

\subsection{Front End Electronics}

The Front End Electronics are still to a large part undefined.
The base of it is forseen to be built around the TOFPET ASIC by PETsys.
This however would not deliver a complete system on its own.
Components of the FEE that would still need to be integrated are the 
\begin{itemize}
    \item Temperature sensor control,
    \item Calibration LED control.
\end{itemize}

The temperature sensor can easily be controlled and communicated with by a micro controller or via an FPGA.
The internal FPGA of the TOFPET FEB/D would be a suitable candidate to take over these duties if it can spare some computing capacity.
The FEB/D already communicates with the temperature sensors in the FEB/A, so it should be possible to extend this functionality towards additional sensors along the \railboard .

The LED control seems fairly simple but depends on the implementation of the calibration LED's.
If they are simply installed as is, the electronics are only required to provide a sharp electric pulse to drive the LED's for a split second to produce a short burst of light.

\subsection[]{\railboard\ Split}

The question whether the split of the \railboard\ is necessary remains to be answered.
It was considered an option in order to be able to use RO4003C a low loss PCB substrate material which does not come in sheets larger than about \SI{1}{m}.
The reduced signal amplitude losses are negligible however since the dielectric constant is smaller than for FR-4 we are able to use less material reducing the material budget.
It remains to be seen if the connectors between the front and back part of the \railboard\ has any adverse effects on the time resolution of the detector.
If no such negative effects manifest the board with RO4003C is to be preferred.

\subsection{\sensorboard Behavior}

While performing measurements with the flexible \sensorboard s some unexpected behavior was observed.
The signal pulse shapes can be distorted depending on how bent the flex board is, with reflections on the line, some of which are stronger than others.
These measurements however were done with unoptimized board dimension, since the influence of bending the flex boards and the resulting layout change might affect the characteristic impedance.
%Another board with proper dimensions for a characteristic impedance of \SI{50}{\ohm} has been made. 
Test on whether the behavior persists and how to mitigate the distortions are still pending and subject to further investigation.

\end{document}